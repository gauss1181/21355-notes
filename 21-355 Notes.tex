\documentclass[11pt]{article}

\usepackage{amsmath}
\usepackage{amssymb}
\usepackage{fancyhdr}

\usepackage{hyperref}
\hypersetup{
    colorlinks,
    citecolor=black,
    filecolor=black,
    linkcolor=black,
    urlcolor=black
}

\oddsidemargin0cm
\topmargin-2cm
\textwidth16.5cm
\textheight23.5cm

\setlength{\parindent}{0pt}
\setlength{\parskip}{5pt plus 1pt}
 
\pagestyle{fancyplain}
\rhead{\fancyplain{}{\today}}
\chead{\fancyplain{}{21-355 Notes}}

\newcommand{\qeq}{\stackrel{?}{=}}

\begin{document}

\begin{titlepage}
	\vspace*{\fill}
	\begin{center}
		{\Huge 21-355: Real Analysis 1}\\[0.5cm]
		{\Large Carnegie Mellon University}\\[0.3cm]
		{\Large Professor Ian Tice - Fall 2013}\\[2cm]
		 Project \LaTeX'd by Ivan Wang
	\end{center}
	\vspace*{\fill}
\end{titlepage}

\newpage

\tableofcontents
\newpage

\section{The Number Systems}

\subsection{The Natural Numbers}

\textbf{Theorem} (existence of $\mathbb{N}$): There exists a set $\mathbb{N}$ satisfying the following properties, known as the Peano Axioms:
\begin{quote}
	\begin{enumerate}
	\item[\bf PA1] $0 \in \mathbb{N}$
	
	\item [\bf PA2] There exists a function $S: \mathbb{N} \to \mathbb{N}$ called the successor function. In particular, $S(n) \in \mathbb{N}$.
	
	\item[\bf PA3] $\forall n \in \mathbb{N}.\; S(n) \neq 0$
	
	\item[\bf PA4] $S(n) = S(m) \implies n = m$ ($S$ is injective, one-to-one)

	\item[\bf PA5] [Axiom of Induction] Let $P(n)$ be a property associated to each $n \in \mathbb{N}$.\\
	If $P(0)$ is true, and $P(n) \implies P(S(n))$, then $P(n)$ is true $\forall n \in \mathbb{N}$.
	\end{enumerate}
\end{quote}

\textbf{Definition}: \textbf{PA1} $\implies 0 \in \mathbb{N}$. \textbf{PA2} $\implies S(0) \in \mathbb{N}$.
\begin{quote}\vspace{-0.3cm}
	Define $1 = S(0), 2 = S(1), 3 = S(2)$, etc.

	\textbf{PA2} guarantees that $\{0, 1, 2, \cdots\} \subseteq \mathbb{N}$.

	\textbf{PA3} prevents ``wraparound": no successor can map to a ``negative" number.

	\textbf{PA4} prevents ``stagnation": the cycle does not terminate.
\end{quote}

\textbf{Theorem}: $\mathbb{N} = \{0, 1, 2, \cdots\}$

\emph{Proof}: We know that $\{0, 1, 2, \cdots\} \subseteq \mathbb{N}$, so it suffices to prove that $\mathbb{N} \subseteq \{0, 1, 2, \cdots\}$.

\begin{quote}\vspace{-0.3cm}
	Let $P(n)$ denote the proposition that $n \in \{0, 1, 2, \cdots\}$. Clearly $P(0)$ is true.

	Suppose $P(n)$ is true; then $n \in \{0, 1, 2, \cdots\} \implies S(n) \in \{0, 1, 2, \cdots\}$ by construction.\\
	Hence, $P(S(n))$ is true. By induction, \textbf{PA5} guarantees that $P(n)$ is true $\forall n \in \mathbb{N}$.

	It follows that $\mathbb{N} \subseteq \{0, 1, 2, \cdots\}$.
\end{quote}

\textbf{Definition}: For any $m \in \mathbb{N}$, we define $0 + m = m$.\\
Then if $n + m$ is defined for $n \in \mathbb{N}$, we set $S(n) + m = S(n + m)$.

\textbf{Proposition} (Properties of Addition):
\begin{quote}\vspace{-0.3cm}
	\begin{enumerate}
	\item $\forall n \in \mathbb{N}.\; n + 0 = n$ \hspace{4.2cm} (0 is the additive identity)
	\item $\forall m,n \in \mathbb{N}.\; n + S(m) = S(n + m)$
	\item $\forall m,n \in \mathbb{N}.\; m + n = n + m$ \hspace{2.8cm} (commutativity)
	\item $\forall k,m,n \in \mathbb{N}.\; k + (m+n) = (k+m) + n$ \hspace{0.5cm} (associativity)
	\item $\forall k,m,n \in \mathbb{N}.\; n + k = n + m \implies k = m$ \hspace{0.5cm} (cancelation)
	\end{enumerate}
\end{quote}

\emph{Proof}:
\begin{quote}\vspace{-0.3cm}
	\begin{enumerate}
	\item Let $P(n)$ be $n + 0 = n$.\\ $P(0)$ is true because $0 + 0 = 0$ by definition.\\
	Note $P(n) \implies S(n) + 0 = S(n + 0) = S(n)$, so $P(S(n))$ is true. By induction, (1) is true.

	\item Fix $m \in \mathbb{N}$. Let $P(n)$ denote $n + S(m) = S(n+m)$.\\
	$P(0)$ is true because $0 + S(m) = S(m) = S(0 + m)$.\\
	$P(n) \implies S(n) + S(m) = S(n + S(m)) = S(S(n+m)) = S(S(n) + m)$, so $P(S(n))$ is true. By induction, since $m \in \mathbb{N}$ was arbitrary, (2) is true.

	\item Let $m$ be fixed and $P(n)$ denote $n + m = m + n$.\\
	$P(0)$ is true since $0 + m = m$ by definition, and $m + 0 = m$ by 1, so $0 + m = m = m + 0$.\\
	Suppose $P(n)$; then $S(n) + m = S(n + m) = S(m + n) = m + S(n)$, so $P(S(n))$ is true. By induction and arbitrary choice of $m$, (3) is true.

	\item Fix $k,m \in \mathbb{N}$ and let $P(n)$ denote $k + (m + n) = (k + m) + n$.\\
	$P(0)$ is true as $k + (m + 0) = k + m = (k + m) + 0$.\\
	Suppose $P(n)$; then $k + (m + S(n)) = k + S(m + n) = S(k + (m + n)) = S(k + m) + n = (k + m) + S(n)$ by (2). By induction and arbitrary choice, (4) is true.

	\item Fix $m,n \in \mathbb{N}$ and let $P(k)$ denote proposition 5.\\
	$P(0)$ is true because $n + 0 = n = n + m \implies m = 0 \implies k = m$.\\
	Suppose $P(k)$; also, suppose $m + S(k) = n + S(k)$. Then $S(m+k) = m + S(k) = n + S(k) = S(n + k) \implies m + k = n + k \implies m = n$ (by 4). By the axiom of induction, (5) is true.
	\end{enumerate}
\end{quote}

\subsubsection{Positivity}

\textbf{Definition}: We say that $n \in \mathbb{N}$ is \emph{positive} if $n \neq 0$.

\textbf{Proposition} (Properties of Positivity):
\begin{quote}\vspace{-0.3cm}
	\begin{enumerate}
	\item $\forall n, m \in \mathbb{N}$, if $m$ is positive, then $m + n$ is positive.
	\item $\forall n, m \in \mathbb{N}$, if $m + n = 0$, then $m = n = 0$.
	\item $\forall n \in \mathbb{N}$, if $n$ is positive, then there exists a unique $m \in \mathbb{N}$ such that $n = S(m)$.
	\end{enumerate}
\end{quote}

\subsubsection{Order}

\textbf{Definition}: For all $m, n \in \mathbb{N}$, $m \leq n$ or $n \geq m$ iff $n = m + p$ for some $p \in \mathbb{N}$.\\
\hspace{0.5cm}$m < n$ or $n > m$ iff $m \leq n \land m \neq n$. The relation $\leq$ provides what is called an \emph{order} on $\mathbb{N}$.

\textbf{Proposition} (Properties of Order):
\begin{quote}\vspace{-0.3cm}
	Let $j,k,m,n \in \mathbb{N}$. Then:
	\begin{enumerate}
	\item $n \geq n$ (reflexitivity)
	\item $m \leq n \land k \leq m \implies k \leq n$ (transitivity)
	\item $m \geq n \land m \leq n \implies m = n$ (anti-symmetry)
	\item $j \leq k \land m \leq n \implies j + m \leq k + n$ (order preservation)
	\item $m < n \iff S(m) \leq n$
	\item $m < n \iff n = m + p$ for some positive $p \in \mathbb{N}$.
	\item $n \geq m \iff S(n) > m$
	\item $n = 0 \oplus 0 < n$
	\end{enumerate}
\end{quote}

\textbf{Theorem} (Trichotomy of Order): Let $m, n \in \mathbb{N}$. Then exactly one of the following is true:
\begin{displaymath}
	m < n \hspace{0.5cm} \oplus \hspace{0.5cm} m = n \hspace{0.5cm} \oplus \hspace{0.5cm} m > n
\end{displaymath}

\emph{Proof}: Show that no two can be true simultaneously (by definition of $<$ and $>$), and then at least one must be true (by induction on $n$).

\subsubsection{Multiplication}

\textbf{Definition}: Fix $m \in \mathbb{N}$. Define $0 \cdot m = 0$. Now, if $n \cdot m$ is defined for some $n \in \mathbb{N}$, we define $S(n) \cdot m = n \cdot m + m$.

\textbf{Proposition} (Properties of Multiplication):
\begin{quote}\vspace{-0.3cm}
	Fix $k,m,n \in \mathbb{N}$. Then:
	\begin{enumerate}
	\item $m \cdot n = n \cdot m$ (commutativity)
	\item $m,n$ are positive $\implies mn$ is positive
	\item $m \cdot n = 0 \iff m = 0 \lor n = 0$ (no zero divisors)
	\item $k \cdot (m \cdot n) = (k \cdot m) \cdot n$ (associativity)
	\item $k \cdot m = k \cdot n \land k$ is positive $\implies m = n$ (cancelation)
	\item $k \cdot (m + n) = (m + n) \cdot k = k \cdot m + k \cdot n$ (distributivity)
	\item $m < n \land k \leq l \land k,l$ are positive $\implies m \cdot k < n \cdot l$
	\end{enumerate}
\end{quote}

\subsection{The Integers}

Consider the following relation on the set $\mathbb{N} \times \mathbb{N}$:
\begin{displaymath}
	(m,n) \simeq (m', n') \iff m + n' = m' + n
\end{displaymath}

\textbf{Lemma}: $\simeq$ is an equivalence relation.

\emph{Proof}:
\begin{quote}\vspace{-1cm}
	\item Reflexivity: $m + n = m + n \implies (m,n) \simeq (m,n)$
	\item Symmetry: $(m,n) \simeq (m', n') \implies m + n' = m' + n \implies m' + n = m + n' \implies (m', n') \simeq (m, n)$
	\item Transitivity: Suppose $(m,n) \simeq (m', n') \land (m', n') \simeq (m'', n'')$. Then:
	\begin{align*}
		& m+n' = m' + n \land m' + n'' = m'' + n'\\
		\implies & m + n'' = m'' + n\\
		\implies & (m,n) \simeq (m'',n'')
	\end{align*}
\end{quote}

\textbf{Definition}: Write the \emph{equivalence class} of $(m,n)$ as $[(m,n)] = \{(p,q) \;|\; (p,q) \simeq (m,n)\}$.\\
Define the \emph{integers} $\mathbb{Z} = \{[(m,n)]\}$.

\textbf{Lemma}: Suppose $(m,n) \simeq (m', n'), (p,q) \simeq (p',q')$. Then:
\begin{quote}\vspace{-0.3cm}
	\begin{enumerate}
	\item $(m+p, n+q) \simeq (m'+p', n'+q')$
	\item $(mp+nq, mq+np) \simeq (m'p'+n'q', m'q'+n'p')$
	\end{enumerate}
\end{quote}

\emph{Proof}: Consider equalities $(a): m+n' = m'+n$ and $(b): p+q' = p'+q$ (by definition of $\simeq$).

Using linear combinations of $(a)$ and $(b)$, we derive the two rules of the lemma:
\begin{quote}\vspace{-0.3cm}
	\begin{enumerate}
	\item $(a) + (b)$
	\item $(a)(p' + q') + (b)(m + n)$
	\end{enumerate}
\end{quote}

\textbf{Definition}: Let $[(m,n)], [(p,q)] \in \mathbb{Z}$. Then:
\begin{quote}\vspace{-0.3cm}
	\begin{enumerate}
	\item $[(m,n)] + [(p,q)] = [(m+p, n+q)]$ (addition of integers)
	\item $[(m,n)] \cdot [(p,q)] = [(mp+nq, mq+np)]$ (multiplication of integers)
	\end{enumerate}
\end{quote}
By the lemma, these are well-defined operations.

Note that for all $m,n \in \mathbb{N}$:
\begin{align*}
[(m,0)] = [(n,0)] &\iff m+0 = n+0 \iff m = n\\
[(m,0)] + [(n,0)] &= [(m+n, 0)]\\
[(m,0)] \cdot [(n,0)] &= [(mn, 0)]
\end{align*}
As such, the set $\{[(n,0)] \;|\; n \in \mathbb{N}\} \subseteq \mathbb{Z}$ behaves exactly like a copy of $\mathbb{N}$.

\textbf{Definition}: For $n \in \mathbb{N}$ we set $n \in \mathbb{Z}$ to be $n := [(n,0)]$.

\hspace{2cm}For $x = [(m,n)] \in \mathbb{Z}$ we define $-x = [(n,m)]$.

\subsubsection{Properties of Integers}

(We can see that every integer $x \in \mathbb{Z}$ can be represented as $x := m - n$ where $x = [(m,n)]$.)

\textbf{Theorem}: Every $x \in \mathbb{Z}$ satisfies exactly one of the following:
\begin{quote}\vspace{-0.3cm}
	\begin{enumerate}
	\item $x = n$ for some $n \in \mathbb{N} \backslash \{0\}$
	\item $x = 0$
	\item $x = -n$ for some $n \in \mathbb{N} \backslash \{0\}$
	\end{enumerate}
\end{quote}
\emph{Proof}: Write $x = [(p,q)]$ for some $p,q \in \mathbb{N}$. By trichotomy of order on $\mathbb{N}$ we know that $p < q$ or $p = q$ or $p > q$. Each of these correlates to one of the three properties.

\textbf{Corollary}: $\mathbb{Z} = \{0,1,2, \ldots \} \cup \{-1, -2, -3, \ldots \}$

\subsubsection{Algebraic Properties}

\textbf{Proposition}: Let $x,y,z \in \mathbb{Z}$. Then the following hold:
\begin{quote}\vspace{-0.3cm}
	\begin{enumerate}
	\item $x + y = y + x$
	\item $x + (y+z) = (x+y) + z$
	\item $x + 0 = 0 + x = x$
	\item $x + (-x) = (-x) + x = 0$
	\item $xy = yx$
	\item $(xy)z = x(yz)$
	\item $x \cdot 1 = 1 \cdot x = x$
	\item $x(y+z) = xy + xz$
	\end{enumerate}
\end{quote}
\textbf{Definition}: Define $x-y = x+ (-y)$. The usual properties hold.

\textbf{Definition}: For $x,y \in \mathbb{Z}$, we say $x \leq y$ or $y \geq x$ if $y - x = n$ for some $n \in \mathbb{N}$.\\
We say $x < y$ if $x \leq y \land x \neq y$.

\subsection{The Rationals and Ordered Fields}

Let a relation on $\mathbb{Z} \times (\mathbb{Z} \backslash \{0\})$ be given by $(m,n) \simeq (m',n') \iff mn' = m'n$.

\textbf{Lemma}: $\simeq$ is an equivalence relation. Proof follows from properties of $\mathbb{Z}$.

\textbf{Definition}: $\mathbb{Q} = \{[(m,n)]\}$
\begin{quote}\vspace{-0.3cm}
	\begin{enumerate}
	\item $[(m,n)] + [(p,q)] = [(mq+np, nq)]$ (addition)
	\item $[(m,n)] \cdot [(p,q)] = [(mp, nq)]$ (multiplication)
	\item $-[(m,n)] = [(-m, n)]$ (negation)
	\item If $m \neq 0$ we set $[(m,n)]^{-1} = [(n,m)]$
	\end{enumerate}
\end{quote}
Remark: the heuristic here is that $\frac{m}{n} = [(m,n)]$.

\textbf{Definition}: If $m \in \mathbb{Z}$, we write $m = [(m,1)] \in \mathbb{Q}$; and thus $\mathbb{N} \subset \mathbb{Z} \subset \mathbb{Q}$.
\begin{quote}\vspace{-0.3cm}
	\begin{enumerate}
	\item For $x,y \in \mathbb{Q}$, we define $x-y = x+(-y) \in \mathbb{Q}$
	\item For $x,y \in \mathbb{Q}, y \neq 0$ we define $\frac{x}{y} = x(y)^{-1}$. This is well defined because $y=0 \iff y = [(0,n)]$.\end{enumerate}
\end{quote}
\textbf{Proposition}: $\mathbb{Q} = \{\frac{m}{n} \;|\; m,n \in \mathbb{Z}, n \neq 0\}$.

We define and propose the trichotomy of order on $\mathbb{Q}$, as per the integers.

\subsubsection{Fields and Orders}

\textbf{Definition}: A field is a set $\mathbb{F}$ endowed with two binary operations, $+, \cdot$, satisfying the following axioms:
\begin{quote}\vspace{-0.3cm}
	\begin{enumerate}
	\item[(A1, M1)] $\forall x,y \in \mathbb{F}.\; x+y \in \mathbb{F}, xy \in \mathbb{F}$ (closure)
	\item[(A2, M2)] $\forall x,y \in \mathbb{F}.\; x + y = y + x,\; xy = yx$ (commutativity)
	\item[(A3, M3)] $\forall x,y,z \in \mathbb{F}.\; x + (y+z) = (x+y) + z,\; x(yz) = (xy)z$ (associativity)
	\item[(A4, M4)] $\exists (0,1) \in \mathbb{F}.\; \forall x \in \mathbb{F}.\; 0 + x = x + 0 = x,\; 1 \cdot x = x \cdot 1 = x$ (identity)
	\item[(A5, M5)] $\forall x \in \mathbb{F}.\; \exists (-x).\; x + (-x) = 0;\; \exists x^{-1} \in \mathbb{F}.\; xx^{-1} = x^{-1}x = 1$ (inverse)
	\item[(D1)] $\forall x,y,z \in \mathbb{F}.\; x(y+z) = xy + xz$ (distributivity)
	\end{enumerate}
\end{quote}

\emph{Remark}: Field must have at least 2 elements (0, 1) by (A/M4). To prove field, must prove 5 properties of addition and multiplication (closure, symmetry, associativity, identity, inverse) as well as distributivity.

\textbf{Definition}: Let $E$ be a set; an \emph{order} on $E$ is a relation $<$ satisfying the following:
\begin{quote}\vspace{-0.3cm}
	\begin{enumerate}
	\item $\forall x,y \in E$ exactly one of the following is true: $x < y$ or $x = y$ or $y < x$ (trichotomy)
	\item $\forall x,y,z \in E$, $x < y \land y < z \implies x < z$ (transitivity)
	\end{enumerate}
\end{quote}

\textbf{Definition}: Let $\mathbb{F}$ be a field. Then we define $x - y = x + (-y)$ and $\frac{x}{y} = xy^{-1}$ (for $y \neq 0$).

\textbf{Theorem}: $\mathbb{Q}$ is an ordered field with order $<$.

\emph{Proof}: Follows from definitions and properties of $\mathbb{Z}$.


\subsection{Problems with $\mathbb{Q}$}

\textbf{Theorem}: There does not exist a $q \in \mathbb{Q}$ such that $q^2 = 2$.

\emph{Proof}: Suppose not; i.e. there does exist such a $q \in \mathbb{Q}$.
\begin{quote}\vspace{-0.3cm}
Consider the set $S(q) = \{n \in \mathbb{N}^+ \;|\; q = \frac{m}{n}$ for some $m \in \mathbb{Z}\}$. Cleary $|S(q)| > 0$. Then the well-ordering principle implies that $\exists !n \in S(q).\; n = \min S(q)$.

Since $n \in S(q)$, we know that $q = \frac{m}{n}$ for some $m \in \mathbb{Z}$. Then $q^2 = (\frac{m}{n})^2 = \frac{m^2}{n^2} \implies m^2 = 2n^2 \implies m^2$ is even. We claim that $m$ is also even (proof is exercise to reader).

Then $\exists l \in \mathbb{Z}.\; m = 2l$. Then $4l^2 = (2l)^2 = m^2 = 2n^2 \implies n^2 = 2l^2 \implies n^2$ is even $\implies n$ is even $\implies n = 2p$ for some $p \in \mathbb{N}^+$.

Hence $q = \frac{m}{n} = \frac{2l}{2p} = \frac{l}{p} \implies p \in S(q)$. But clearly $p < n$, which contradicts the fact that $n$ is the minimal element. By contradiction, the theorem must be true.
\end{quote}

\subsubsection{Bounds (Infimum and Supremum)}

Informally, $\mathbb{Q}$ has ``holes":

\textbf{Definition}: Let $E$ be an ordered set with order $<$.
\begin{quote}\vspace{-0.3cm}
	\begin{enumerate}
	\item We say $A \subseteq E$ is bounded above iff $\exists x \in E.\; \forall a \in A.\; a \leq x$. We say $x$ is an upper bound of $A$.
	\item We say $A \subseteq E$ is bounded below iff $\exists x \in E.\; \forall a \in A.\; x \leq a$. We say $x$ is a lower bound of $A$.
	\item We say $A \subseteq E$ is bounded iff it's bounded above and below.
	\item We say $x$ is a minimum of $A$ iff $x \in A$ and $x$ is a lower bound of $A$.
	\item We say $x$ is a maximum of $A$ iff $x \in A$ and $x$ is an upper bound of $A$.
	\end{enumerate}
\end{quote}
\emph{Remark}: If a min or max exists, then it is unique.

\textbf{Definition}: Let $E$ be an ordered set and $A \subseteq E$.
\begin{quote}\vspace{-0.3cm}
	\begin{enumerate}
	\item We say $x \in E$ is the least upper bound (\emph{supremum}) of $A$, written $x = \sup A$, iff $x$ is an upper bound of $A$ and $y \in E$ is an upper bound of $A \implies x \leq y$.
	\item We say $x \in E$ is the greatest lower bound (\emph{infimum}) of $A$, written $x = \inf A$, iff $x$ is a lower bound of $A$ and $y \in E$ is a lower bound of $A \implies y \leq x$.
	\end{enumerate}
\end{quote}
\emph{Remark}: If $x = \min(A)$, then $x = \inf (A)$. If $x = \max(A)$, then $x = \sup (A)$. But the converse is false; some sets have a supremum but no maximum, others a infimum but no minimum.

\textbf{Definition}: Let $\mathbb{F}$ be an ordered field. We say that $\mathbb{F}$ has the \emph{least upper bound property} iff every $\varnothing \neq A \subseteq \mathbb{F}$ that is bounded above has a least upper bound.

\textbf{Theorem}: $\mathbb{Q}$ does not satisfy the least upper bound property.

\emph{Proof}: Consider the set $A = \{x \in \mathbb{Q} \;|\; x > 0, x^2 \leq 2\}$.
\begin{quote}\vspace{-0.3cm}
Note that $0 < 1 = 1^2 \leq 2 \implies 1 \in A$, so $A$ is non-empty. Also, $2 \leq 4 = 2^2$ implies $(x \in A \implies 0 < x^2 < 2 < 2^2) \implies x < 2$. Then 2 is an upper bound of $A$.

Assume for sake of contradiction that $\mathbb{Q}$ has the least upper bound property. Then $A$ has a supremum. Let $x = \sup A \in \mathbb{Q}$ and write $x = \frac{p}{q}$ for $p,q \in \mathbb{Z}$.

By trichotomy, $x^2 < 2$ or $x^2 = 2$ or $x^2 > 2$. We know $x^2 \neq 2$.

\textbf{Case 1}: Suppose $x^2 < 2$. Then for any $n \in \mathbb{N}^+$ we have $(\frac{p}{q} + \frac{1}{n})^2 = \frac{p^2}{q^2} + \frac{2p}{qn} + \frac{1}{n^2} \leq \frac{p^2}{q^2} + \frac{1}{n}(\frac{2p+q}{q})$. From algebra, we derive $(\frac{p}{q} + \frac{1}{n})^2 < 2$ for some $n \in \mathbb{N}^+$.

Cleary $x > 0$ since otherwise $x \leq 0 < 1 \in A$. Hence $0 < x = \frac{p}{q} < \frac{p}{q} + \frac{1}{n} \in A$. But then $x$ is not an upper bound $\implies$ contradiction.

\textbf{Case 2}: Suppose $x^2 > 2$. Considering $(\frac{p}{q} - \frac{1}{n})^2 > 2$ and using the same logic as before, we can choose $n$ large enough such that $\frac{p}{q} - \frac{1}{n}$ is an upper bound of $A$. But $\frac{p}{q} - \frac{1}{n} < \frac{p}{q} = x$, which contradicts the fact that $x = \sup A$.

As all cases are false, we contradict trichotomy, and hence $\mathbb{Q}$ cannot have the least upper bound property.
\end{quote}

\subsection{The Real Numbers}

We now construct an ordered field satisfying the least upper bound property using $\mathbb{Q}$.

\textbf{Definition}: We say $\mathbb{Q}$ is Archimedean iff $\forall (x \in \mathbb{Q}).\; x > 0 \implies \exists (n \in \mathbb{N}).\; x < n$.

\textbf{Lemma}: If $\mathbb{Q}$ is Archimedean, then $\forall (p < q \in \mathbb{Q}).\; \exists (r \in \mathbb{Q}).\; p < r < q$.\\
(Proofs in HW 2.)

\subsubsection{Defining the Real Numbers: Dedekind Cuts}

\textbf{Definition}: We say that $\mathcal{C} \in \mathcal{P}(\mathbb{Q})$ is a \emph{cut} (Dedekind cut) iff the following hold:
\begin{quote}\vspace{-0.3cm}
	\begin{enumerate}
	\item[(C1)] $\varnothing \neq \mathcal{C}, \mathcal{C} \neq \mathbb{Q}$
	\item[(C2)] If $p \in \mathcal{C}$ and $q \in \mathbb{Q}$ with $q < p$, then $q \in \mathcal{C}$.
	\item[(C3)] If $p \in \mathcal{C}$, $\exists (r \in \mathbb{Q}).\; p < r \land r \in \mathcal{C}$.
	\end{enumerate}
\end{quote}

\textbf{Lemma}: Suppose $\mathcal{C}$ is a cut. Then:
\begin{quote}\vspace{-0.3cm}
	\begin{enumerate}
	\item $p \in \mathcal{C}, q \notin \mathcal{C} \implies p < q$
	\item$r \notin \mathcal{C}, r < s \implies s \notin \mathcal{C}$
	\item $\mathcal{C}$ is bounded above
	\end{enumerate}
\end{quote}

\textbf{Lemma}: Let $q \in \mathbb{Q}$. Then $\{p \in \mathbb{Q} \;|\; p < q\}$ is a cut.

\emph{Proof}: Call the set $\mathcal{C}$. We prove the 3 properties of a cut:
\begin{quote}\vspace{-0.3cm}
	\begin{enumerate}
	\item[(C1)] $q - 1 \in \mathcal{C} \implies \mathcal{C} \neq \varnothing$; $q + 1 \notin \mathcal{C} \implies \mathcal{C} \neq \mathbb{Q}$.
	\item[(C2)] If $p \in \mathcal{C}$ and $r \in \mathbb{Q}$ such that $r < p$, then $r < p < q \implies r < q \implies r \in \mathcal{C}$.
	\item[(C3)] Let $p \in \mathcal{C}$ where $p < q$. Since $\mathbb{Q}$ is Archimedean, $\exists (r \in \mathbb{Q}).\; p < r < q \implies r \in \mathcal{C}$.
	\end{enumerate}
\end{quote}

\textbf{Definition}: Given $q \in \mathbb{Q}$ we write $\mathcal{C}_q = \{p \in \mathbb{Q} \;|\; p < q\}$. By the above lemma, $\mathcal{C}_q$ is a cut.

\textbf{Definition}: We write $\mathbb{R} = \{\mathcal{C} \in \mathcal{P}(\mathbb{Q}) \;|\; \mathcal{C} \text{ is a cut}\} \neq \varnothing$.

\textbf{Lemma}: The following hold:
\begin{quote}\vspace{-0.3cm}
	\begin{enumerate}
	\item $\forall \mathcal{A}, \mathcal{B} \in \mathbb{R}$, exactly one of the following holds: $\mathcal{A} \subset \mathcal{B}, \mathcal{A} = \mathcal{B}, \mathcal{B} \subseteq \mathcal{A}$.
	\item $\forall \mathcal{A}, \mathcal{B}, \mathcal{C} \in \mathbb{R}$, $\mathcal{A} \subset \mathcal{B} \land \mathcal{B} \subseteq \mathcal{C} \implies \mathcal{A} \subset \mathcal{C}$.
	\end{enumerate}
\end{quote}

\textbf{Definition}: If $\mathcal{A}, \mathcal{B} \in \mathbb{R}$ we say that $\mathcal{A} < \mathcal{B} \iff \mathcal{A} \subset \mathcal{B}$, and $\mathcal{A} \leq \mathcal{B} \iff \mathcal{A} \subseteq \mathcal{B}$. This defines an order on $\mathbb{R}$ by the above lemma.

\subsubsection{Defining the Real Numbers: The Least Upper Bound Property}

\textbf{Lemma}: Suppose $\varnothing \neq E \subseteq \mathbb{R}$ is bounded above. Then $\mathcal{B} := \bigcup_{\mathcal{A} \in E} \mathcal{A} \in \mathbb{R}$.

\textbf{Theorem}: $\mathbb{R}$ satisfies the least upper bound property.

\emph{Proof}: Let $\varnothing \neq E \subseteq \mathbb{R}$ be bounded above and set $\mathcal{B} = \bigcup_{\mathcal{A} \in E} \mathcal{A} \in \mathbb{R}$. We claim $\mathcal{B} = \sup E$.
\begin{quote}\vspace{-0.3cm}
First, we show that $\mathcal{B}$ is an upper bound of $E$. Let $\mathcal{A} \in E$. Then $\mathcal{A} \subseteq \mathcal{B} \implies \mathcal{A} \leq \mathcal{B}$ (by definition). This is true for all $\mathcal{A} \in E$, so $\mathcal{B}$ is an upper bound.

We claim that for $\mathcal{C} \in \mathbb{R}.\; \mathcal{C} < \mathcal{B} \implies \mathcal{C}$ is not an upper bound of $E$. If $\mathcal{C} < \mathcal{B}$, then $\mathcal{C} \subset \mathcal{B}$. This implies $\exists b \in \mathcal{B}.\; b \notin \mathcal{C} \implies \exists(\mathcal{A} \in E).\; b \in \mathcal{A} \land b \notin \mathcal{C}$. Then $\mathcal{A} > \mathcal{C}$ since otherwise $\mathcal{A} \subseteq \mathcal{C} \implies b \in \mathcal{C}, b \notin \mathcal{C}$. Hence $\mathcal{C} < \mathcal{A}$ and $\mathcal{C}$ is not an upper bound of $E$.

By the contrapositive: if $\mathcal{C}$ is an upper bound, $\mathcal{C} \geq \mathcal{B}$. Thus, $\mathcal{B}$ is the least upper bound, and the theorem holds.
\end{quote}

\subsubsection{Defining the Real Numbers: Addition}

\textbf{Definition}: Given $\mathcal{A}, \mathcal{B} \in \mathbb{R}$, set $\mathcal{A} + \mathcal{B} = \{a+b \;|\; a \in \mathcal{A}, b \in \mathcal{B}\}$.

\textbf{Lemma}: If $\mathcal{A}, \mathcal{B} \in \mathbb{R}$, then $\mathcal{A} + \mathcal{B} \in \mathbb{R}$.

\textbf{Theorem}: Define $-\mathcal{A} = \{q \in \mathbb{Q} \;|\; \exists(p >q).\; -p \notin \mathcal{A}\}$. Then $\mathbb{R}, +, 0_\mathbb{R} = \mathcal{C}_0 = \{p \in \mathbb{Q} \;|\; p < 0\}$ satisfy the field axioms.

\emph{Proof}:
\begin{quote}\vspace{-0.3cm}
	\begin{enumerate}
	\item[(A1)] $\mathcal{A} + \mathcal{B} \in \mathbb{R}$ by previous lemma.
	\item[(A2)] $\mathcal{A} + \mathcal{B} = \{a + b\} = \{b + a\} = \mathcal{B} + \mathcal{A}$.
	\item[(A3)] $\mathcal{A} + (\mathcal{B} + \mathcal{C}) = \{a + (b+c)\} = \{(a+b) + c\} = (\mathcal{A} + \mathcal{B}) + \mathcal{C}$.
	\item[(A4)] Show $\forall \mathcal{A} \in \mathbb{R}.\; 0_\mathbb{R} + \mathcal{A} = \mathcal{A}$.
	\item[(A5)] Show that $-\mathcal{A} \in \mathbb{R}$, then $\mathcal{A} + (-\mathcal{A}) = 0_\mathbb{R}$ using Archimedean property.
	\end{enumerate}
\end{quote}
\textbf{Theorem} (Ordered Field): Let $\mathcal{A}, \mathcal{B}, \mathcal{C} \in \mathbb{R}$. If $\mathcal{A} < \mathcal{B}$ then $\mathcal{A} + \mathcal{C} < \mathcal{B} + \mathcal{C}$.

\emph{Proof}: It's trivial to see that $\mathcal{A} \subseteq \mathcal{B} \implies \mathcal{A} + \mathcal{C} \subseteq \mathcal{B} + \mathcal{C} \implies \mathcal{A} + \mathcal{C} \leq \mathcal{B} + \mathcal{C}$.
\begin{quote}\vspace{-0.3cm}
If $\mathcal{A} + \mathcal{C} = \mathcal{B} + \mathcal{C}$, we can add $- \mathcal{C}$ to both sides and use the last theorem to see that $\mathcal{A} = \mathcal{B}$, a contradiction. Hence, $\mathcal{A} + \mathcal{C} < \mathcal{B} + \mathcal{C}$.
\end{quote}

\subsubsection{Defining the Real Numbers: Multiplication}

\textbf{Lemma}: Let $\mathcal{A}, \mathcal{B} \in \mathbb{R}$, $\mathcal{A}, \mathcal{B} > 0_\mathbb{R}$. Then $\mathcal{C} = \{q \in \mathbb{Q} \;|\; q \leq 0\} \cup \{a \cdot b \;|\; a \in \mathcal{A}, b \in \mathcal{B}, a, b > 0\} \in \mathbb{R}$.

\emph{Proof}:
\begin{quote}\vspace{-0.3cm}
	\begin{enumerate}
	\item[(C1)] $0 \in \mathcal{C} \implies \mathcal{C} \neq \varnothing$. $\mathcal{A}, \mathcal{B}$ are bounded above by, say $M_1, M_2$, so $M_1 \cdot M_2 + 1 \notin \mathcal{C}$ and $\mathcal{C} \neq \mathbb{Q}$.
	\item[(C2)] Let $p \in \mathcal{C}$ and $q < p$. If $q \leq 0$ then $q \in \mathcal{C}$ by definition. If $q > 0$ then $0 < q < p$, but then $0 < p \implies p = a \cdot b$ for $a \in \mathcal{A}, b \in \mathcal{B}, a, b > 0$. Then $0 < q < a \cdot b \implies \frac{q}{a} < b \implies 0 < \frac{q}{a} \in \mathcal{B}$. Then $q = a(\frac{q}{a}) \in \mathcal{C}$.
	\item[(C3)] Let $p \in \mathcal{C}$. If $p \leq 0$ then any $a \cdot b$ with $a \in \mathcal{A}, b \in \mathcal{B}, a,b > 0$ satisfies $p < a \cdot b \in \mathcal{C}$, so $r = a \cdot b$ is the desired element of $\mathcal{C}$. However, if $p > 0$, then $p = a \cdot b$ for $a \in \mathcal{A}, b \in \mathcal{B}, a,b > 0$. Choose $s \in \mathcal{A}$ such that $a < s, t \in \mathcal{B}$ such that $t > b$. Then $p = a \cdot b < s \cdot t \in \mathcal{S}$, so $r = s \cdot t$ proves the claim.
	\end{enumerate}
\end{quote}

\textbf{Definition of Multiplication}: Let $\mathcal{A}, \mathcal{B} \in \mathbb{R}$.
\begin{quote}\vspace{-0.3cm}
	\begin{enumerate}
	\item If $\mathcal{A} > 0, \mathcal{B} > 0$ we set $\mathcal{A} \cdot \mathcal{B} = \{q \in \mathbb{Q} \;|\; q \leq 0\} \cup \{a \cdot b \;|\; a \in \mathcal{A}, b \in \mathcal{B}, a, b > 0\} \in \mathbb{R}$.
	\item If $\mathcal{A} = 0$ or $\mathcal{B} = 0$, we set $\mathcal{A} \cdot \mathcal{B} = 0_\mathbb{R}$.
	\item If $\mathcal{A} > 0$ and $\mathcal{B} < 0$, let $\mathcal{A} \cdot \mathcal{B} = - (\mathcal{A} \cdot (-\mathcal{B}))$.
	\item If $\mathcal{A} < 0$ and $\mathcal{B} > 0$, let $\mathcal{A} \cdot \mathcal{B} = -((-\mathcal{A}) \cdot \mathcal{B})$.
	\item If $\mathcal{A} < 0$ and $\mathcal{B} < 0$, let $\mathcal{A} \cdot \mathcal{B} = (-\mathcal{A}) \cdot (-\mathcal{B})$.
	\end{enumerate}
\end{quote}

\textbf{Theorem}: $\mathbb{R}, \cdot$ satisfies (M1-M5) with $1_\mathbb{R} = \mathcal{C}_1$, and\\
$\mathcal{A} > 0 \implies \mathcal{A}^{-1} = \{q \in \mathbb{Q} \;|\; q \leq 0\} \cup \{q \in \mathbb{Q} \;|\; q > 0, \exists p > q.\; p^{-1} \notin \mathcal{A}\} \in \mathbb{R}$;\\
$\mathcal{A} < 0 \implies \mathcal{A}^{-1} = -(-\mathcal{A})^{-1}$.

\emph{Proof}: HW3 (similar to addition).

\textbf{Theorem}: If $\mathcal{A}, \mathcal{B} > 0$, then $\mathcal{A} \cdot \mathcal{B} > 0$.

\emph{Proof}: By definition $\mathcal{C}_0 \subseteq \mathcal{A} \cdot \mathcal{B} \implies 0 \leq \mathcal{A} \cdot \mathcal{B}$. Equality is impossible since $\mathcal{A}, \mathcal{B} > 0$.

\subsubsection{Defining the Real Numbers: Distributivity}

\textbf{Theorem}: Let $\mathcal{A}, \mathcal{B}, \mathcal{C} \in \mathbb{R}$. Then $\mathcal{A} \cdot (\mathcal{B} + \mathcal{C}) = \mathcal{A} \cdot \mathcal{B} + \mathcal{A} \cdot \mathcal{C}$.

\emph{Proof}: We prove the case where all are positive. The other cases are in HW.
\begin{quote}\vspace{-0.3cm}
Let $p \in \mathcal{A}(\mathcal{B} + \mathcal{C})$. If $p \leq 0$ then $p \in \mathcal{A} \cdot \mathcal{B} + \mathcal{A} \cdot \mathcal{B}$ is trivial (both products contain the interval less than 0).

If $p > 0$, $p = a(b+c)$ for $a \in \mathcal{A}, b \in \mathcal{B}, c \in \mathcal{C}$ for $a > 0, b + c > 0$.

Regardless of sign of $b$ or $c$, $a \cdot b \in \mathcal{A} \cdot \mathcal{B}, a \cdot c \in \mathcal{A} \cdot \mathcal{C}$. Hence $p = a(b+c) = a \cdot b + a \cdot c \in \mathcal{A} \cdot \mathcal{B} + \mathcal{A} \cdot \mathcal{C}$. So $\mathcal{A}(\mathcal{B} + \mathcal{C}) \subseteq \mathcal{A} \cdot \mathcal{B} + \mathcal{A} \cdot \mathcal{C}$.

Finally, we show the converse is true; let $p \in \mathcal{A} \cdot \mathcal{B} + \mathcal{A} \cdot \mathcal{C} \implies p = r+s$ for $r \in \mathcal{A} \cdot \mathcal{B}, s \in \mathcal{A} \cdot \mathcal{C}$. Case on positivity of $p,r,s$ to show $p \in \mathcal{A}(\mathcal{B} + \mathcal{C})$.
\end{quote}

\subsubsection{Defining the Real Numbers: Archimedean}

\textbf{Theorem}: For $p,q \in \mathbb{Q}$, the following are true:
\begin{quote}\vspace{-0.3cm}
	\begin{enumerate}
	\item $\mathcal{C}_{p+q} = \mathcal{C}_p + \mathcal{C}_q$
	\item $\mathcal{C}_{-p} = -\mathcal{C}_p$
	\item $\mathcal{C}_{pq} = \mathcal{C}_p \mathcal{C}_q$
	\item If $p \neq 0$ then $\mathcal{C}_{p^{-1}} = (\mathcal{C}_p)^{-1}$
	\item $p < q \in \mathbb{Q} \iff \mathcal{C}_p < \mathcal{C}_q \in \mathbb{R}$
	\end{enumerate}
\end{quote}
\emph{Proof}: HW.

\textbf{Definition}: For $q \in \mathbb{Q}$ we say $\mathcal{C}_q \in \mathbb{R}$. Then $\mathbb{Q} \subseteq \mathbb{R}$.

\textbf{Theorem}: There exists an ordered field satisfying the least upper bound property; $\mathbb{R}$ is unique (for any ordered field $\mathbb{F}$ satisfying these properties, $\mathbb{F} = \mathbb{R}$ up to isomorphism; and $\mathbb{R}$ is Archimedean.

\emph{Proof}: The basic assertion is Steps (0)-(4). Step (5) proves 1, Step (6) proves 3.

\subsection{Properties of $\mathbb{R}$}

Notation: think of $\mathbb{R}$ as numbers, not cut notation.

\textbf{Proposition}: $\mathbb{R}$ satisfies the following:

\textbf{Theorem}: For $p,q \in \mathbb{Q}$, the following are true:
\begin{quote}\vspace{-0.3cm}
	\begin{enumerate}
	\item $\mathbb{R}$ is Archimedean: $\forall x \in \mathbb{R}, x > 0.\; \exists n \in \mathbb{N}.\; x < n$
	\item $\mathbb{N} \subset \mathbb{R}$ is not bounded above
	\item $\inf\{\frac{1}{n} \;|\; n \in \mathbb{N}, n \geq 1\} = 0$
	\item $\forall x \in \mathbb{R}$ the set $B(x) = \{m \in \mathbb{Z} \;|\; x < m\}$ has a minimum in $\mathbb{Z}$.
	\item $\forall x,y \in \mathbb{R}, x < y.\; \exists q \in \mathbb{Q}.\; x < q < y$
	\end{enumerate}
\end{quote}

\emph{Remarks:}
\begin{quote}\vspace{-0.3cm}
	\begin{enumerate}
	\item (5) is interpreted as ``the density of $\mathbb{Q} \subseteq \mathbb{R}$". Any element $x \in \mathbb{R}$ can be approximated to arbitrary accuracy by elements of $\mathbb{Q}$.
	\item (4) allows us to define the integer part of any $x \in \mathbb{R}$. We can set $\lfloor x \rfloor = \min B(x) - 1 \in \mathbb{Z}$. Then $\lfloor x \rfloor \leq x < \lfloor x \rfloor + 1$.
	\end{enumerate}
\end{quote}

Next we show that $\mathbb{R}$ does not have the ``holes" we saw in $\mathbb{Q}$.

\textbf{Theorem}: Let $x \in \mathbb{R}$ satisfy $x > 0$ and $n \in \mathbb{N}, n \geq 1$. Then $\exists ! y \in \mathbb{R}.\; y > 0 \land y^n = x$.

\emph{Proof}: The case $n = 1$ is trivial so assume $n \geq 2$.
\begin{quote}\vspace{-0.3cm}
Set $E = \{z \in \mathbb{R} \;|\; z > 0 \land z^n < x\}$. We want to show $E \neq \varnothing$ and is bounded above. Set $t = \frac{x}{1+x}$; then $0 < t < 1$ and $t < x$. Hence $0 < t^n < t < x$, and so $t \in E$ and $E \neq \varnothing$.

Set $s = 1+x$. Then $1 < s \land x < s \implies x < s < s^n$; so if $z \in E$ then $z^n < x < s^n \implies z < s$. Then $s$ is an upper bound of $E$.

By least upper bound property, $\exists y \in \mathbb{R}.\; y = \sup E$. Since $t \in E$, $0 < t < y$, so $y > 0$. We claim that $y^n < x$ and $y^n > x$ are both impossible (proof is exercise), so $y^n = x$.
\end{quote}

\textbf{Definition}: Let $n \geq 1$; for $x \in \mathbb{R}, x > 0,$ we write $x^{\frac{1}{n}} = y$ where $y^n = x$. We set $0^{\frac{1}{n}} = 0$.


\subsubsection{Absolute Value}

For $x \in \mathbb{R}$, we define the function $|\cdot| : \mathbb{R} \to \{r \in \mathbb{R} \;|\; r \geq 0\}$:
\[
 |x| =
  \begin{cases}
   x & \text{if } x > 0\\
   0 & \text{if } x = 0\\
  -x & \text{if } x < 0
  \end{cases}
\]

\textbf{Proposition} (Properties of $|\cdot|$):
\begin{quote}
	\begin{enumerate}
	\item $\forall x \in \mathbb{R}.\; |x| \geq 0$ and $|x| = 0 \iff x = 0$
	\item $\forall x,y \in \mathbb{R}.\; |x| < y \iff -y < x < y$
	\item $\forall x,y \in \mathbb{R}.\; |xy| = |x||y|$
	\item $\forall x,y \in \mathbb{R}.\; |x+y| \leq |x| + |y|$ (Triangle Inequality)
	\item $\forall x,y \in \mathbb{R}.\; ||x|-|y|| \leq |x-y|$
	\end{enumerate}
\end{quote}

\section{Sequences}

Let $E$ be a set. Then we may define a sequence $\{a_n\}_{n=l}^\infty \subseteq E$ as the set of values $a_n \equiv a(n)$ for some $l \in \mathbb{Z}$ and some function $a : \{n \in \mathbb{Z} \;|\; n \geq l\} \to E$.

\subsection{Convergence and Bounds}

\textbf{Definition}: We say a sequence $\{a_n\}_{n=l}^{\infty} \subseteq \mathbb{R}$ converges to $a \in \mathbb{R}$, i.e. $a_n \to a$ as $n \to \infty$ or $\lim_{n \to \infty} a_n = a$, if for every $0 < \epsilon \in \mathbb{R}$, there exists $N \in \{m \in \mathbb{Z} \;|\; m \geq l\}$ such that $n \geq N \implies |a_n - a| < \epsilon$.

\textbf{Definition}: We say a sequence $\{a_n\}_{n=l}^{\infty} \subseteq \mathbb{R}$ is bounded iff. $\exists M \in \mathbb{R}, M > 0.\; |a_n| < M \;(\forall n \geq l)$.

\textbf{Lemma}: If a sequence converges, then it is bounded.

\textbf{Definition}: Given $\{a_n\}, \{b_n\} \subseteq \mathbb{R}$ we define $\{a_n + b_n\} \subseteq \mathbb{R}$ to be the sequence whose elements are $a_n + b_n$. We similary define $\{ca_n\}$ for a fixed $c \in \mathbb{R}$, $\{a_nb_n\}$, and $\{a_n/b_n\}$ where $b_n \neq 0, n \geq l$.

\textbf{Theorem} (algebra of convergence): Let $\{a_n\}, \{b_n\} \subseteq \mathbb{R}, c \in \mathbb{R}$, and assume that $a_n \to a, b_n \to b$ as $n \to \infty$. Then the following hold:
\begin{quote}\vspace{-0.3cm}
	\begin{enumerate}
	\item $a_n + b_n \to a + b$ as $n \to \infty$
	\item $ca_n \to ca$ as $n \to \infty$
	\item $a_nb_n \to ab$ as $n \to \infty$
	\item If $b_n \neq 0$ and $b \neq 0$, then $a_n/b_n \to a/b$ as $n \to \infty$.
	\end{enumerate}
\end{quote}

\emph{Proof}: (1), (2) are in next week's HW.
\begin{quote}\vspace{-0.3cm}
(3): Note that $|a_nb_n - ab| = |a_nb_n - ab_n + ab_n - ab| \leq |a_nb_n - ab_n| + |ab_n - ab| = |b_n||a_n-a| + |a||b_n - b|$. Since $b_n \to b$ we know that $\exists M > 0.\; |b_n| < M (\forall n \geq l)$.

Let $\epsilon > 0$. Since $a_n \to a$ and $b_n \to b$ we may choose $N_1$ such that $n \geq N_1 \implies |a_n - a| < \frac{\epsilon}{2M}$; and $N_2$ where $n \geq N_2 \implies |b_n - b| < \frac{\epsilon}{2(1+|a|)}$.

Then set $N = \max(N_1, N_2)$. So if $n \geq N$ we know that $|a_nb_n - ab| \leq |b_n||a_n-a| + |a||b_n-b| < M|a_n - a| + |a||b_n-b| < M \cdot \frac{\epsilon}{2M} + |a| \cdot \frac{\epsilon}{2(1+|a|)} < \frac{\epsilon}{2} +\frac{\epsilon}{2} = \epsilon$.

Since $\epsilon$ was arbitrary, we deduce that $a_nb_n \to ab$.\\

(4): We know $|\frac{a_n}{b_n} - \frac{a}{b}| = |\frac{a_nb - ab_n}{b_nb}| = |\frac{a_nb - ab + ab - ab_n}{b_nb}| \leq \frac{|a_nb - ab|}{|b_n||b|} + \frac{|ab - ab_n|}{|b||b_n|} = \frac{|a_n-a|}{|b_n|} + \frac{|a|}{|b||b_n|} |b_n - b|$.

Let $\epsilon > 0$. Since $b_n \to b \neq 0$ we know that $\exists N_1$ such that $n \geq N_1 \implies |b_n - b| < \frac{|b|}{2}$. Then $n \geq N \implies 0 < |b| = |b - b_n + b_n| \leq |b-b_n| + |b_n| < \frac{|b|}{2} + |b_n| \implies 0 < \frac{|b|}{2} \leq |b_n| \implies 0 < \frac{1}{|b_n|} < \frac{2}{|b|}$.

Similarly, $a_n \to a \implies \exists N_2 .\; (n \geq N_2 \implies |a_n - a| < \frac{\epsilon}{4} |b|$; and\\
$b_n \to b \implies \exists N_3.\; (n \geq N_3 \implies |b_n - b| < \frac{\epsilon |b|^2}{4(1+|a|)}$.

Set $N = \max(N_1, N_2, N_3)$. Then $n \geq N \implies |\frac{a_n}{b_n} - \frac{a}{b}| \leq \frac{|a_n-a|}{|b_n|} + \frac{|a|}{|b_n||b|} |b_n - b| < \frac{2}{|b| |a_n - a|} + \frac{2|a|}{|b|^2} |b_n - b| < \frac{\epsilon}{2} + \frac{\epsilon}{2} \frac{|a|}{1+|a|} < \epsilon$.

Since $\epsilon > 0$ was arbitrary, we deduce $\frac{a_n}{b_n} \to \frac{a}{b}$ as $n \to \infty$.
\end{quote}

\textbf{Lemma}: Let $\{a_n\}_{n=l}^\infty$ converge to $a \in \mathbb{R}$. Then $\forall \epsilon > 0.\; \exists N.\; m, n \geq N \implies |a_n - a_m| < \epsilon$.

\textbf{Definition}: We say $\{a_n\}_{n=l}^\infty \subseteq \mathbb{R}$ is \emph{Cauchy} iff $\forall \epsilon > 0.\; \exists N.\; m,n \geq N \implies |a_n - a_m| < \epsilon$.

\textbf{Lemma}: If $\{a_n\}$ is Cauchy, then it's bounded.

\emph{Proof}: Let $\epsilon = 1$. Then $\exists N.\; m,n \geq N \implies |a_m - a_n| < 1$. Then $n \geq N \implies |a_n - a_N| < 1 \implies |a_n| < |a_n - a_N| + |a_N| < 1 + |a_N|$. Set $M = \max(1+ |a_N|, k)$, where $k = \max\{|a_l|, \ldots, |a_{N-1}|\}$. Then $|a_n| < M (\forall n \geq l)$, and $\{a_n\}$ is bounded.

\textbf{Theorem}: Let $\{a_n\} \subseteq \mathbb{R}$. Then $\{a_n\}$ converges $\iff \{a_n\}$ is Cauchy.

\emph{Proof}: $\implies$ is covered by 2nd-previous lemma. We show the converse:
\begin{quote}\vspace{-0.3cm}
Suppose $\{a_n\}$ is Cauchy. Then $|a_n| < M (\forall n \geq l)$ by the last lemma.

Set $E = \{x \in \mathbb{R} \;|\; \exists N.\; n \geq N \implies x < a_n\}$. Note that $-M < a_n (\forall n \geq l)$, and so $-M \in E$ and $E \neq \varnothing$.

Also, $x \in E \implies \exists N_x.\; n \geq N_x \implies x < a_n < M$, and so $M$ is an upper bound of $E$.\\
By the least upper bound property of $\mathbb{R}$, $\exists a = \sup E \in \mathbb{R}$. We claim that $a_n \to a$ as $n \to \infty$.

Let $\epsilon > 0$. Then since $\{a_n\}$ is Cauchy, $\exists N.\; m,n \geq N \implies |a_n - a_m| < \frac{\epsilon}{2}$. In particular, $|a_n - a_N| < \frac{\epsilon}{2}$ when $n \geq N$. Then $n \geq N \implies a_N - \frac{\epsilon}{2} < a_n \implies a_N - \frac{\epsilon}{2} \in E \implies a_N - \frac{\epsilon}{2} \leq a$.

If $x \in E$, then $\exists E_x.\; (n \geq N_x \implies x < a_n < a_N + \frac{\epsilon}{2})$. Hence $a_N + \frac{\epsilon}{2}$ is an upper bound of $E \implies a \leq a_N + \frac{\epsilon}{2}$. Then $|a - a_N| < \frac{\epsilon}{2}$.

But if $n \geq N$, then $|a_n - a| \leq |a_n - a_N| + |a_N - a| < \frac{\epsilon}{2} + \frac{\epsilon}{2} = \epsilon$. Hence $a_n \to a$.
\end{quote}

\subsubsection{Squeeze Lemma}

\textbf{Lemma}: Let $\{a_n\}_{n=l}^\infty, \{b_n\}, \{c_n\} \subseteq \mathbb{R}$ and suppose that $a_n \to a, c_n \to a$ as $n \to \infty$. If $\exists k \geq l$ such that $a_n \leq b_n \leq c_n (\forall n \geq k)$, then $b_n \to a$ as $n \to \infty$.

Examples:
\begin{enumerate}\vspace{-0.3cm}
\item Suppose $a_n \to 0$ and $\{b_n\}$ is bounded, i.e. $|b_n| \leq M (\forall n \geq l)$. Then $|a_n b_n| = |a_n| |b_n| \leq |a_n| M$. But $c_n \to 0 \iff |c_n| \to 0$. Then $0 \leq |a_n b_n| \leq |a_n| M$, both sides of which go to 0; and by the squeeze lemma, $|a_n b_n| \to 0 \implies a_n b_n \to 0$.

\item  Fix $k \in \mathbb{N}$ with $k \geq 1$. Set $a_n = \frac{1}{n^k}, n \geq 1$. Then $0 \leq \frac{1}{n^k} \leq \frac{1}{n}$, and by squeeze lemma $\frac{1}{n^k} \to 0$.

\item Fix $k \in \mathbb{N}$ with $k \geq 2$. Let $a_n = \frac{1}{k^n}, n \geq 0$. We know $\forall n \in \mathbb{N}.\; n \leq k^n$ (proof by induction). Then $0 \leq \frac{1}{k^n} \leq \frac{1}{n}$, and by squeeze $\frac{1}{k^n} \to 0$.
\end{enumerate}

\subsection{Monotonicity and limsup, liminf}

\textbf{Definition}: Let $\{a_n\}_{n=l}^\infty \subseteq \mathbb{R}$. We say $\{a_n\}$ is:
\begin{quote}\vspace{-0.3cm}
	\begin{enumerate}
	\item increasing iff. $a_n < a_{n+1} (\forall n \geq l)$,
	\item non-decreasing iff. $a_n \leq a_{n+1} (\forall n \geq l)$,
	\item decreasing iff. $a_{n+1} < a_n (\forall n \geq l)$,
	\item non-increasing iff. $a_{n+1} \leq a_n (\forall n \geq l)$.
	\end{enumerate}
\end{quote}
We say $\{a_n\}$ is \emph{monotone} iff. it is either non-increasing or non-decreasing.

\emph{Remark}: increasing $\implies$ non-decreasing, decreasing $\implies$ non-increasing.

\textbf{Theorem}: Suppose that $\{a_n\}_{n=l}^\infty \subseteq \mathbb{R}$ is monotone. Then $\{a_n\}$ is bounded iff $\{a_n\}$ is convergent.

\emph{Proof}: $\Longleftarrow$ is done in a previous lemma.
\begin{quote}
$\implies$: We'll prove when the sequence is non-decreasing (other case handled by similar argument).

Set $E = \{a_n \;|\; n \geq l\} \subseteq \mathbb{R}$. Clearly $E \neq \varnothing$. Also, since $\{a_n\}$ is bounded, $E$ is as well (in particular above). By least upper bound property of $\mathbb{R}$, $\exists a = \sup(E) \in \mathbb{R}$. We claim that $a = \lim_{n \to \infty} a_n$.

Let $\epsilon > 0$. Since $a = \sup(E)$ we know that $a- \epsilon$ is not an upper bound of $E$; hence $\exists (N \geq l).\; a - \epsilon < a_N$. Also, since the sequence is non-decreasing, $a_n \leq a_{n+1} (\forall n \geq l)$, and so $n \geq N \implies a_N \leq a_n$. Then $n \geq N \implies a - \epsilon < a_N \leq a_n \leq a$ because $a$ is an upper bound of $E$.

So $n \geq N \implies -\epsilon < a_n - a \leq 0 \implies |a_n - a| < \epsilon$. Since $\epsilon > 0$ was arbitrary, we deduce that $a_n \to a$ as $n \to \infty$.
\end{quote}

\textbf{Lemma}: Suppose that $\{a_n\}$ is bounded. Set $S_m = \sup \{a_n \;|\; n \geq m\}$ and $I_m = \inf \{a_n \;|\; n \geq m\}$. Then $S_m, I_m \in \mathbb{R}$ are well-defined $\forall m \geq l$; $\{S_m\}$ is non-increasing; and $\{I_m\}$ is non-decreasing. Both sequences are bounded.

\textbf{Definition}: Suppose $\{a_n\} \subseteq \mathbb{R}$ is bounded. We set $\lim_{n \to \infty} \sup a_n = \lim_{m \to \infty} S_m \in \mathbb{R}$ and $\lim_{n \to \infty} \inf a_n = \lim_{m \to \infty} I_m \in \mathbb{R}$. Both limits exist by the lemma and previous theorem. We know that $\lim_{n \to \infty} \inf a_n \leq \lim_{n \to \infty} \sup a_n$ from HW.

\subsection{Subsequences}

\textbf{Definition}: Let $\phi: \{n \in \mathbb{Z} \;|\; n \geq l\} \to \{n \in \mathbb{Z}\;|\; n \geq l\}$ be order preserving (increasing), i.e. $m < n$ then $\phi(m) < \phi(n)$. Let $\{a_n\}_{l=k}^\infty \subseteq \mathbb{R}$ be a sequence. We say $\{a_{\phi(k)}\}_{k=l}^\infty$ is a \emph{subsequence} of $\{a_n\}$.

\emph{Remarks}:
\begin{quote}\vspace{-0.3cm}
	\begin{enumerate}
	\item $\phi(k) = k$ is order preserving, so every sequence is a subsequence of itself.
	\item Not every $a_n$ has to be in the subsequence $\{a_{\phi(k)}\}$.

	For example, if $l = 0$ then $\phi(k) = 2k$ is order preserving. In this case $a_n, n$ odd does not appear in the subsequence $\{a_{\phi(k)}\}$.

	\item We will often write $n_k = \phi(k)$ to simplify notation, so $\{a_{n_k}\}$ denotes a subsequence.
	\item From HW1, we know $k \leq \phi(k) \; (\forall k \geq l)$.
	\end{enumerate}
\end{quote}

\emph{Proposition}: Suppose $\{a_n\}$ satisfies $a_n \to a \in \mathbb{R}$ as $n \to \infty$. Then any subsequence of $\{a_n\}$ also converges to $a$.

\emph{Proof}:
\begin{quote}\vspace{-0.3cm}
Let $\{a_{\phi(k)}\}$ be a subsequence of $\{a_n\}$. Let $\epsilon > 0$. Since $a_n \to a$ as $n \to \infty$, we know $\exists N \geq l.\; n \geq N \implies |a_n - a| < \epsilon$. We claim $\exists K \geq l.\; k \geq K \implies \phi(k) \geq N$.

If not, then $\phi(k) < N (\forall k \geq l)$; but $k \leq \phi(k) < N (\forall k \geq l)$ is a contradiction. Then the claim is true, and $k \geq K \implies \phi(k) \geq N \implies |a_{\phi(k)} - a| < \epsilon$. Since $\epsilon > 0$ was arbitrary, we deduce $\{a_{\phi(k)}\} \to a$ as $k \to \infty$.
\end{quote}

\emph{Remark}: Converse fails. Example: $a_n = (-1)^n$; $a_{2n} = +1 \to +1$, but $a_{2n+1} = -1 \to -1$.

\subsubsection{Limsup Theorem}

\textbf{Theorem}: Let $\{a_n\} \subseteq \mathbb{R}$ be bounded. The following hold:
\begin{quote}\vspace{-0.3cm}
	\begin{enumerate}
	\item Every subsequence of $\{a_n\}$ is bounded.
	\item If $\{a_{n_k}\}$ is a subsequence, then $\lim_{k \to \infty} \sup a_{n_k} \leq \limsup_{n \to \infty} a_n$.
	\item If $\{a_{n_k}\}$ is a subsequence, then $\lim_{n \to \infty} \inf a_n \leq \liminf_{n \to \infty} a_{n_k}$.
	\item There exists a subsequence $\{a_{n_k}\}$ such that $\lim_{k \to \infty} a_{n_k} = \limsup_{n \to \infty} a_n$.
	\item There exists a subsequence $\{a_{n_k}\}$ such that $\lim_{k \to \infty} a_{n_k} = \liminf_{n \to \infty} a_n$ ($\neq$ (4)).
	\end{enumerate}
\end{quote}

\emph{Proof}:
\begin{quote}\vspace{-0.3cm}
	\begin{enumerate}
	\item Trivial.
	\item Since $k \leq \phi(k)$, $\{a_{\phi(n)} \;|\; n \geq k\} \subseteq \{a_n \;|\; n \geq k\}$ for every order-preserving $\phi$. Hence $S_k = \sup \{a_{\phi(n)}\} \;|\; n \geq k\} \subseteq \sup \{a_n \;|\; n \geq k\} = T_k$. But:\\
	$\limsup_{n \to \infty} a_{\phi(n)} = \lim_{k \to \infty} \sup \{a_{\phi(n)} \;|\; n \geq k\} \leq \limsup_{k \to \infty} \{a_n \;|\; n \geq k\} = \limsup_{n \to \infty} a_n$.
	\item Similar to (2); exercise to reader.
	\item Too lazy to \LaTeX; exercise to reader. %% Insert proof %%
	\item Exercise to reader.
	\end{enumerate}
\end{quote}

\textbf{Theorem}: Suppose $\{a_n\} \subseteq \mathbb{R}$; the following are equivalent:
\begin{quote}\vspace{-0.3cm}
	\begin{enumerate}
	\item $a_n \to a$ as $n \to \infty$
	\item $\{a_n\}$ is bounded, and every convergent subsequence converges to $a$.
	\item $\{a_n\}$ is bounded, and $\limsup_{n \to \infty} a_n = \liminf_{n \to \infty} a_n$.
	\end{enumerate}
\end{quote}
\emph{Proof}: $(1) \implies (2)$ proven already.
\begin{quote}\vspace{-0.3cm}
$(2) \implies (3)$\\
Limsup theorem (4,5) $\implies \exists \{a_{\phi(k)}\}, \{a_{\gamma(k)}\}$ subsequences such that $a_{\phi(k)} \to \limsup_{n \to \infty} a_n, a_{\gamma(k)} \to \liminf_{n \to \infty} a_n$ as $k \to \infty$. By (2) the limits must agree.

$(3) \implies (1)$\\
%% Insert proof %%

\end{quote}

\textbf{Theorem} (Bolzano-Weierstrass): If $\{a_n\} \subseteq \mathbb{R}$ is bounded then there exists a convergent subsequence. Proof from (4) or (5) of Limsup Theorem.

\subsection{Special Sequences}

\textbf{Definition}: Given $a_n \in \mathbb{R}$ for $0 \leq k \leq n, n \in \mathbb{N}$ we define $\sum_{k=0}^n a_n = a_0 + a_1 + \cdots + a_n$.

\textbf{Lemma} (Binomial Theorem): Let $x,y \in \mathbb{R}$ and $n \in \mathbb{N}$. Then $(x+y)^n = \sum\limits_{k=0}^n {n \choose k} x^k y^{n-k}$, where ${n \choose k} := \frac{n!}{k!(n-k)!} \in \mathbb{N}$.

\textbf{Theorem}: In the following assuming that $n \geq 1$:
\begin{quote}\vspace{-0.3cm}
	\begin{enumerate}
	\item Let $x \in \mathbb{R}, x > 0$. Then $a_n = \frac{1}{n^x} \to 0$ as $n \to \infty$.
	\item Let $x \in \mathbb{R}, x > 0$. Then $a_n = x^{1/n} \to 1$ as $n \to \infty$.
	\item Let $a_n = n^{1/n}$; then $a_n \to 1$ as $n \to \infty$.
	\item Let $a,x \in \mathbb{R}, x > 0$. Then $\frac{n^a}{(1+x)^a} \to 0$ as $n \to \infty$.
	\item Let $x \in \mathbb{R}, |x| < 1$. Then $a_n = x^n \to 0$ as $n \to \infty$.
	\end{enumerate}
\end{quote}

\section{Series}

\textbf{Definition}: Let $\{a_n\}_{n=l}^\infty \subseteq \mathbb{R}$; for $p < q$ we write $\sum_{n=p}^q a_n = (a_p + \cdots + a_q)$.
\begin{quote}\vspace{-0.3cm}
	\begin{enumerate}
	\item We define, for each $n \geq l$, $S_n = \sum\limits_{k=l}^n a_k \in \mathbb{R}$ to be the $n^\text{th}$ partial sum of $\{a_n\}_{n=l}^\infty$.
	\item If $\exists s \in \mathbb{R}.\; S_n \to s$ as $n \to \infty$, then $\sum_{n=l}^\infty a_n = s$. We say the "infinite series" $\sum_{n=l}^\infty a_n$ converges.
	\item If the series does not converge, it diverges.
	\end{enumerate}
\end{quote}

\textbf{Examples}
\begin{enumerate}\vspace{-0.3cm}
	\item Let $a_n = x^n$ for $n \geq 0, x \in \mathbb{R}$. Then $S_n = \sum_{k=0}^n x^k$. Notice that $(1-x)S_n = \sum_{k=0}^n x^k - \sum_{k=0}^n x^{k+1} = \sum_{k=0}^n x^k - \sum_{k=1}^{n+1} x^k = 1-x^{n+1}$.

So $S_n = \sum_{k=0}^n x^k = (\frac{1-x^{n+1}}{1-x})$. If $|x| < 1$ then $S_n \to \frac{1}{1-x}$ by special seq (5).

	\item Suppose $\{b_n\}_{n=0}^\infty \subseteq \mathbb{R}$ where $b_n \to b$ as $n \to \infty$. Set $a_n = b_{n+1}-b_n$ for $n \geq 0$. Then the series $\sum_{n=0}^\infty a_n$ converges and in fact $\sum_{n=0}^\infty = b - b_0$.
\end{enumerate}

\subsection{Convergence Results}

We develop tool sthat will let us deduce the convergence of a series without knowing its value.

\textbf{Theorem}: Suppose $\sum_{n=l}^\infty a_n$ converges. Then $a_n \to 0$ as $n \to \infty$.

\emph{Proof}: Notice that $a_n = S_n - S_{n-1}$ and so $\lim_{n \to \infty} a_n = \lim_{n \to \infty} (S_n - S_{n-1}) = S - S = 0$.

\textbf{Corollary}: $\sum_{n=0}^\infty (-1)^n$ and $\sum_{n=0}^\infty n$ diverge, as neither sequences converge to 0.

\textbf{Corollary}: The series $\sum_{n=0}^\infty x^n$ converges $\iff |x| < 1$.

\emph{Proof}: $|x| \geq 1 \implies |x^n| = |x|^n \geq 1 (\forall n \in \mathbb{N})$. The converse was proved last time.\\

Next, we provide a characterization of convergence in terms of the size of the ``tails" of the series.

\textbf{Theorem}: $\sum_{n=l}^\infty a_n$ converges $\iff \forall \epsilon > 0.\; \exists N \geq l.\; m \geq k \geq N \implies |\sum_{n=k}^m a_n| < \epsilon$.

\emph{Proof}: $\sum_{n=l}^\infty a_n$ converges $\iff S_k = \sum_{n=l}^k a_n$ converges $\iff \{S_k\}$ is Cauchy.

This is useful in practice because we can guarantee a series converges without knowing its value.\\

\textbf{Theorem}:
\begin{quote}\vspace{-0.3cm}
	\begin{enumerate}
	\item If $\forall n \geq k.\; |a_n| \leq b_n$ for some $k \geq l$, and $\sum\limits_{n=l}^\infty b_n$ converges, then $\sum\limits_{n=l}^\infty a_n$ converges.
	\item If $\forall n \geq k.\; 0 \leq a_n \leq b_n$ for some $k \geq l$, and $\sum\limits_{n=l}^\infty a_n$ diverges, then $\sum\limits_{n=l}^\infty b_n$ diverges.
	\end{enumerate}
\end{quote}

\emph{Proof}: (1) Let $\epsilon > 0$ and prove with previous theorem and induction on triangle inequality. (2) follows from contrapositive.

\textbf{Examples}:
\begin{quote}\vspace{-0.3cm}
	\begin{enumerate}
	\item $\sum_{n=0}^\infty \frac{(-1)^n}{2^n}$ converges because $|\frac{(-1)^n}{2^n}| = \frac{1}{2^n}$ and $\sum_{n=0}^\infty \frac{1}{2^n}$ converges ($\frac{1}{2} < 1$).

	\item Suppose $\sum_{n=0}^\infty a_n$ converges and $a_n \geq 0 \;\forall n \geq 0$. Let $\{b_n\} \subseteq \mathbb{R}$ be bounded, i.e. $|b_n| \leq M \forall n$. Then $|a_nb_n| = |a_n||b_n| \leq Ma_n$. Then $MS_n = M\sum_{k=0}^n a_n = \sum_{k=0}^n M a_n$, so by the theorem, $\sum_{n=0}^\infty a_nb_n$ converges.

	\item $\sum_{n=0}^\infty \frac{(-1)^n}{2^n} \cdot \frac{n!}{n^n} \cdot \frac{3n^2}{4n^2+2}$ converges because the product is bounded.
	\end{enumerate}
\end{quote}

\textbf{Theorem}: Suppose $\forall n \geq l.\; a_n \geq 0$. Then $\sum_{n=l}^\infty a_n$ converges $\iff \{S_n\}_{n=l}^\infty$ is bounded.

\emph{Proof}: Since $a_n \geq 0$, the sequence $S_n = \sum_{k=l}^n a_k$ is non-decreasing: $S_{n+1} = a_{n+1} + S_n \geq S_n$. Since $S_n$ is monotone and converges, it is bounded.

\subsubsection{Cauchy Criterion Theorem}

\textbf{Theorem}: Suppose that $\{a_n\}_{n=1}^\infty \subseteq \mathbb{R}$ satisfies $\forall n \geq l.\; a_n \geq 0$ and $\forall n \geq 1.\; a_{n+1} \leq a_n$. Then $\sum\limits_{n=1}^\infty a_n$ converges $\iff \sum\limits_{n=0}^\infty 2^n a_{2^n}$ converges.

\emph{Proof}:
\begin{quote}\vspace{-0.3cm}
Let $S_n = \sum_{k=1}^n a_k$ and $T_n = \sum_{n=0}^m 2^n a_{2^n}$. Notice that if $m \leq 2^k$ then $S_m = a_1 + a_2 + \cdots + a_{2^k} \leq a_1 + (a_2 + a_3) + \cdots + (a_{2^k} + \cdots + a_{2^{k+1}-1}) \leq a_1 + 2a_2 + \cdots + 2^k a_{2^k} = T_k$.

On the other hand, if $m \geq 2^k$, $S_m \geq a_1 + \cdots + a_{2^k} = a_1 + a_2 + (a_3 + a_4) + \cdots + (a_{2^{k-1}-1} + \cdots + a_{2^k}) \geq \frac{1}{2}a_1 + a_2 + \cdots + 2^{k-1}a_{2^k} = \frac{1}{2}T_k$.\\

Now, if $\sum_{n=0}^\infty 2^n a_{2^n}$ converges, then $T_n \to T$ as $n \to \infty$ and so $S_m \leq \lim_{n \to \infty} T_m = T$, which means $\{S_m\}$ is bounded and $\sum_{n=1}^\infty a_n$ converges.

Similarly, if $\sum_{n=1}^\infty a_n$ converges, then $T_k \leq 2 \lim_{n \to \infty} S_n \implies \{T_k\}$ is bounded $\implies \sum_{n=0}^\infty 2^n a_{2^n}$ converges.
\end{quote}

\textbf{Theorem}: Let $p \in \mathbb{R}$. Then $\sum_{n=1}^\infty \frac{1}{n^p}$ converges $\iff p > 1$.

\emph{Proof}:
\begin{quote}\vspace{-0.3cm}
If $p \leq 0$ the result is trivial since $\frac{1}{n^p} \geq 1$ (the sequences converges to 0). Assume that $p > 0$. Then $\frac{1}{(n+1)^p} \leq \frac{1}{n^p}$, so we can apply the Cauchy criterion:
\begin{displaymath}
\sum_{n=1}^\infty \frac{1}{n^p} \text{ converges} \iff \sum_{n=0}^\infty \frac{2^n}{(2^n)^p} \text{ converges.}
\end{displaymath}
But $\sum_{n=0}^\infty \frac{2^n}{(2^n)^p} = \sum_{n=0}^\infty \frac{1}{(2^{p-1})^n}$, and this series converges $\iff \frac{1}{2^{p-1}} < 1 \iff p > 1$.
\end{quote}

Notice $\sum_{n=1}^\infty \frac{1}{n}$ is divergent, but $\sum_{n=1}^\infty \frac{1}{n^{1+r}}$ converges $\forall r > 0$. To try to find intermediate series, we need the logarithm.

\subsubsection{Logarithm}

\textbf{Definition}: From Supplemental Reading 3, for every $1 < b \in \mathbb{R}$, we define a function\\
$\log_b : \{x \in \mathbb{R} \;|\; x > 0 \} \to \mathbb{R}$ such that
\begin{quote}\vspace{-0.3cm}
	\begin{enumerate}
	\item $b^{\log_b x} = x \;(\forall x > 0)$
	\item $\log_b(1) = 0,\; \log_b b = 1$
	\item $0 < x < y \iff \log_b x < \log_b y$
	\item $\log_b (x^z) = z \log_b (x) \;(\forall x > 0, \forall z \in \mathbb{R})$
	\item $\log_b$ is a bijection
	\item $\lim\limits_{n \to \infty} \frac{\log_b n}{n^r} = 0 \;(\forall r \in \mathbb{R}, r > 0)$
	\end{enumerate}
\end{quote}
Then from (6), for large $n$ and $p > 0$ we know:\\
$n \leq n (\log_b n)^p \leq n \cdot n^p = n^{1+p} \implies \frac{1}{n^{1+p}} \leq \frac{1}{n (\log_b n)^p} \leq \frac{1}{n}$.

So $\frac{1}{n(\log_b n)^p}$ is such an ``intermediate series."

\textbf{Theorem}: Let $b > 1$. $\sum\limits_{n=2}^\infty \frac{1}{n (\log_b n)^p}$ converges $\iff p > 1$. ($n \geq 2 \implies \log_b n > 0$)

\emph{Proof}:
\begin{quote}\vspace{-0.3cm}
$\sum\limits_{n=2}^\infty \frac{1}{n (\log_b n)^p}$ converges $\iff \sum\limits_{n=1}^\infty \frac{2^n}{2^n (\log_b 2^n)^p}$ converges by Cauchy criterion, but\\
$\sum\limits_{n=1}^\infty \frac{1}{(\log_b 2)^p n^p} = \frac{1}{(\log_b 2)^p} \sum\limits_{n=1}^\infty \frac{1}{n^p}$ converges $\iff p > 1$.
\end{quote}
In particular, $\sum\limits_{n=2}^\infty \frac{1}{n \log_b n}$ is divergent.

\subsection{The number $e$}

\textbf{Lemma}: $\sum_{n=0}^\infty \frac{1}{n!}$ converges.

\emph{Proof}: If $n \geq 2$ then:
\begin{align*}
S_n &= \sum_{k=0}^n \frac{1}{k!} = 1 + 1 + \frac{1}{2 \cdot 1} + \cdots + \frac{1}{n(n-1) \cdots 2 \cdot 1}\\
&\leq 1 + 1 + \frac{1}{2} + \frac{1}{2 \cdot 2} + \cdots + \frac{1}{2^{n-1}}\\
&\leq 1 + \sum_{k=0}^\infty \frac{1}{2^k} = 1 + 2 = 3
\end{align*}
Since $S_n$ is increasing and bounded, we know that $\sum_{n=0}^\infty \frac{1}{n!}$ converges.

\textbf{Definition}: We set $e = \sum\limits_{n=0}^\infty \frac{1}{n!}$. Note that $e > 1$.

\textbf{Theorem}: $e = \lim_{n \to \infty} (1+\frac{1}{n})^n$.

\emph{Proof}: Let $S_n = \sum_{k=0}^n \frac{1}{k!}, T_n = (1 + \frac{1}{n})^n$. Then by the Binomial Theorem:
\begin{align*}
T_n &= (1+\frac{1}{n})^n = \sum_{k=0}^n \frac{n!}{k!(n-k)!} \frac{1}{n^k}\\
&= 1 + 1 + \frac{1}{2!} \frac{n(n-1)}{n^2} + \cdots + \frac{1}{n!} \frac{n(n-1) \cdots 1}{n^n}\\
&= 1 + 1 + \frac{1}{2!} (1-\frac{1}{n}) + \frac{1}{3!} (1-\frac{1}{n})(1-\frac{2}{n}) + \cdots + \frac{1}{n!} (1-\frac{1}{n}) \cdots (1-\frac{n-1}{n})\\
&\leq 1 + 1 + \frac{1}{2!} + \cdots + \frac{1}{n!} = S_n
\end{align*}
\begin{quote}\vspace{-0.3cm}
Hence, $\limsup_{n \to \infty} T_n \leq \limsup_{n \to \infty} S_n = \lim_{n \to \infty} S_n = e$.

OTOH, fix $m \in \mathbb{N}$. Then for $n \geq m$:
\begin{align*}
&T_n \geq 1 + 1 + \frac{1}{2!}(1-\frac{1}{n}) + \cdots + \frac{1}{m!} (1-\frac{1}{n}) \cdots (1-\frac{m-1}{n})\\
\implies &\liminf_{n \to \infty} T_n \geq \liminf_{n \to \infty} \text{RHS} \geq 1 + 1 + \frac{1}{2!} \liminf_{n \to \infty} (1-\frac{1}{n}) + \cdots + \frac{1}{m!} \liminf_{n \to \infty} (1-\frac{1}{n} \cdots (1-\frac{m-1}{n})
&= 1 + 1 + \frac{1}{2!} + \cdots + \frac{1}{m!} = S_m
\end{align*}
Then, letting $m \to \infty$, $e = \lim_{m \to \infty} S_m \leq \liminf_{n \to \infty} T_n$.

Thus, $e \leq \liminf_{n \to \infty} T_n \leq \limsup_{n \to \infty} T_n \leq e \implies \lim_{n \to \infty} T_n = e$.
\end{quote}

\textbf{Theorem}: $\forall n \geq 1.\; 0 < e - S_n < \frac{1}{n \cdot n!}$. Also, $e \in \mathbb{R} \backslash \mathbb{Q}$ is irrational.

\emph{Proof}: Since $S_n$ is increasing, $0 < e - S_n$ is clear. The other side can be seen from algebra.
\begin{quote}\vspace{-0.3cm}
Now, suppose $e \in \mathbb{Q}$; then $e = \frac{p}{q}$ for $p, q \in \mathbb{N}, p, q \geq 1$.

Then $0 < q!(e-S_q) < \frac{1}{q} \;(\forall q \geq 1)$. Notice that $q!e = q!\frac{p}{q} = (q-1)!p \in \mathbb{N}$ and\\
$q!(1 + \frac{1}{2!} + \cdots + \frac{1}{q!}) \in \mathbb{N}$.

Hence $q!(e - S_q) \in \mathbb{Z}$; but this yields an integer between 0 and 1, a contradiction. So $e$ is irrational.
\end{quote}
\emph{Remark}: In fact, $e$ is transcendental.

\subsection{More Convergence Results}

%% Missing proofs, examples! %%

\textbf{Theorem (Root Test)}: Suppose $\{a_n\}_{n=l}^\infty \subseteq \mathbb{R}$ and $\{|a_n|^{1/n}\}$ is bounded. Let $0 \leq \alpha = \limsup_{n \to \infty} |a_n|^{1/n}$. Then the following holds:
\begin{quote}\vspace{-0.3cm}
	\begin{enumerate}
	\item If $\alpha < 1$, then $\sum_{n=l}^\infty a_n$ converges.
	\item If $\alpha > 1$, then $\sum_{n=l}^\infty a_n$ diverges.
	\item if $\alpha = 1$, both convergence and divergence are possible.
	\end{enumerate}
\end{quote}

\textbf{Theorem (Ratio Test)}: Let $\{a_n\}_{n=l}^\infty \subseteq \mathbb{R}$. Then $\sum_{n=l}^\infty a_n$:
\begin{quote}\vspace{-0.3cm}
	\begin{enumerate}
	\item converges if $\{|\frac{a_{n+1}}{a_n}|\}_{n=l}^\infty$ is bounded and $\limsup_{n \to \infty} \frac{|a_{n+1}|}{|a_n|} < 1$.
	\item diverges if $\exists k \geq l.\; |a_k| \neq 0$ and $|a_{n+1}| \geq |a_n| (\forall n \geq k)$.
	\end{enumerate}
\end{quote}

\textbf{Lemma (Summation of Parts)}: Let $\{a_n\}_{n=0}^\infty \subseteq \mathbb{R}$ and define:
\[
A_n =
  \begin{cases}
   \sum_{k=0}^n a_k & \text{if } n \geq 0 \\
   0 & \text{if } n = -1
  \end{cases}
\]
Then if $0 \leq p < q$:
\begin{displaymath}
\sum_{n=p}^q a_nb_n = \sum_{n=p}^{q-1} A_n(b_n - b_{n+1}) + A_qb_q - A_{p-1}b_p
\end{displaymath}

\textbf{Theorem (Dirichlet Test)}: Suppose $\{a_n\}_{n=0}^\infty, \{b_n\}_{n=0}^\infty \subseteq \mathbb{R}$ satisfy:
\begin{quote}\vspace{-0.3cm}
	\begin{enumerate}
	\item The sequence $A_n = \sum_{k=0}^n a_k$ is bounded.
	\item $0 \leq b_{n+1} \leq b_n (\forall n \in \mathbb{N}$
	\item $\lim_{n \to \infty} b_n = 0$
	\end{enumerate}
\end{quote}
Then $\sum_{n=0}^\infty a_nb_n$ converges.

\textbf{Corollary (Alternating Series)}: Suppose $0 \leq a_{n+1} \leq a_n, a_n \to 0$ as $n \to \infty$. Then $\sum_{n=l}^\infty (-1)^n a_n$ conveges. Proof follows from Dirichlet Test.

\textbf{Corollary (Abel's Test)}: Suppose $\sum_{n=l}^\infty a_n$ converges, $b_{n+1} \leq b_n (\forall n \geq l)$ and $b_n \to b$ as $n \to \infty$. Then $\sum_{n=l}^\infty a_nb_n$ converges.

\subsection{Algebra of Series}

\textbf{Theorem}: If $A = \sum_{n=l}^\infty a_n, B = \sum_{n=l}^\infty B-N$, then
\begin{align*}
(1) A + B = \sum_{n=l}^\infty (a_n + b_n) \hspace{3cm} (2) cA = \sum_{n=l}^\infty ca_n \;(\forall c \in \mathbb{R})
\end{align*}
\textbf{Theorem}: Suppose $\{a_n\}_{n=0}^\infty, \{b_n\}_{n=0}^\infty \in \mathbb{R}$ satisfy:
\begin{align*}
(1) \sum_{n=0}^\infty |a_n| \text{ converges} \hspace{2cm}
(2) \sum_{n=0}^\infty b_n = B \hspace{2cm}
(3)\; c_n = \sum_{k=0}^n a_kb_{n-k}\text{ for }n \geq 0
\end{align*}
Then $\sum_{n=0}^\infty c_n = A \cdot B$ converges.

\textbf{Definition}: The series $\sum\limits_{n=0}^\infty c_n$, where $c_n = \sum\limits_{k=0}^n a_k b_{n-k}$, is called the \emph{Cauchy product}\\ of the series $\sum\limits_{n=0}^\infty a_n, \sum\limits_{n=0}^\infty b_n$.

\emph{Remark}: If $\sum a_n, \sum b_n$ converge, $\sum c_n$ does not necessarily converge if neither series has convergent absolute values.

\subsection{Absolute Convergence and Rearrangements}

\textbf{Proposition}: If $\sum_{n=l}^\infty |a_n|$ converges, then $\sum_{n=l}^\infty a_n$ converges. Proof is trivial.

\textbf{Definition}: Suppose $\sum_{n=l}^\infty a_n$ converges. If $\sum_{n=l}^\infty |a_n|$ converges, the series converges \emph{absolutely}. If $\sum |a_n|$ diverges, the series is \emph{conditionally convergent}.

\emph{Example}: $\sum_{n=1}^\infty \frac{(-1)^n}{n}$ is conditionally convergent, while $\sum_{n=1}^\infty \frac{(-1)^n}{n^2}$ is absolutely convergent.\\

Let's try to manipulate the series without being careful.
\begin{align*}
\gamma &= \sum_{n=1}^\infty \frac{(-1)^{n+1}}{n} = 1 - \frac{1}{2} + \frac{1}{3} - \frac{1}{4} + \cdots\\
&= \lim_{k \to \infty} (S_k = \sum_{n=0}^k \frac{(-1)^{n+1}}{n}) = \lim_{k \to \infty} (S_{2k} = \sum_{n=0}^{2k} \frac{(-1)^{n+1}}{n})\\
\text{but: } S_{2k} &= (1-\frac{1}{2}) + (\frac{1}{3} - \frac{1}{4} + \cdots + (\frac{1}{2k-1} - \frac{1}{2k}) > 0
\end{align*}
Hence, $\gamma > 0$. But the next step is questionable:
\begin{align*}
2\gamma &= \sum_{n=1}^\infty \frac{(2)(-1)^{n+1}}{n} \qeq \sum_{k=0}^\infty \frac{2}{2k+1} - \sum_{k=1}^\infty \frac{2}{2k}\\
&\qeq \sum_{k=0}^\infty \frac{2}{2k+1} - \sum_{k=1}^\infty \frac{1}{k} = \sum_{k=0}^\infty \frac{1}{2k+1} - \sum_{k=1}^\infty \frac{1}{2k} = \gamma\\
&\implies 2\gamma = \gamma \land \gamma > 0 \hspace{0.5cm} \text{a contradiction!}
\end{align*}
Problem: rearrangement is a delicate issue.

\textbf{Definition}: Let $\gamma : \{m \in \mathbb{Z} \;|\; m \geq l\} \to \{m \in \mathbb{Z} \;|\; m \geq l\}$ be a bijection. The series $\sum_{n=l}^\infty a_{\gamma(n)}$ is called a rearrangement of $\sum_{n=l}^\infty a_n$.

\textbf{Theorem}: If $\sum_{n=l}^\infty a_n$ is absolutely convergent, then every rearrangement converges to $\sum_{n=l}^\infty a_n$.

\emph{Proof}: Let $\epsilon > 0$.
\begin{quote}
Since $\sum_{n=l}^\infty a_n$ converges absolutely, $\exists N \geq l.\; k \geq m \geq N \implies \sum_{n=m}^k |a_n| < \frac{\epsilon}{2}$.

Let $k \to \infty : \sum_{n=m}^\infty |a_n| \leq \frac{\epsilon}{2} < \epsilon$.\\
Now choose $M \geq N$ such that $\{l, l+1, \ldots, N\} \subseteq \{\gamma(l), \gamma(l+1), \ldots, \gamma(M)\}$. Then $m \geq M \implies |\sum_{n=l}^m a_n - \sum_{n=l}^m a_{\gamma(n)}| \leq \sum_{n=N}^\infty |a_n| < \epsilon$.

Hence $\lim_{m \to \infty} (\sum_{n=l}^m a_n - \sum_{n=l}^\infty a_{\gamma(n)}) = 0$ and from this we deduce\\
$\lim_{m \to \infty} \sum_{n=l}^m a_{\gamma(n)} = \lim_{m \to \infty} \sum_{n=l}^m a_n = \sum_{n=l}^\infty a_n$.
\end{quote}

When a series is only conditionally convergent, the situation is vastly worse.

\textbf{Theorem}: Suppose $\sum_{n=0}^\infty a_n$ is conditionally convergent. Let $c \in \mathbb{R}$.\\
There exists a rearrangement (bijection) $\gamma : \mathbb{N} \to \mathbb{N}$ such that $\sum_{n=0}^\infty a_{\gamma(n)} = c$.

\textbf{Lemma}: Suppose $\sum_{n=0}^\infty a_n$ is conditionally convergent and set:
\[
 b_n =
  \begin{cases}
   a_n & \text{if } a_n > 0\\
   0 & \text{if } a_n \leq 0
  \end{cases}\hspace{2cm}
 c_n =
  \begin{cases}
   -a_n & \text{if } a_n < 0\\
   0 & \text{if } a_n \geq 0
  \end{cases}
\]
Then $\sum_{n=0}^\infty b_n$ and $\sum_{n=0}^\infty c_n$ both diverge.

\emph{Proof}: Suppose not; one of the series is convergent. If $\sum b_n$ converges, then $c_n = b_n - a_n \implies \sum c_n = \sum b_n - \sum a_n$; but $|a_n| = b_n + c_n$ and so $\sum |a_n| = \sum b_n + \sum c_n$ is convergent, a contradiction. A similar argument holds if $\sum c_n$ converges.

\textbf{Rearrangement Theorem Proof}:
\begin{quote}\vspace{-0.3cm}
Let $\{a_n^+\}_{n=0}^\infty$ denote the subsequence of $\{b_n \;|\; b_n > 0$ or $b_n = 0 \land a_n = 0\}$. Let $\{a_n^-\}_{n=0}^\infty$ denote the subsequence of $\{c_n \;|\; c_n > 0\}$ (from last lemma). Note:

\begin{enumerate}
\item $a_n^+ \to 0, a_n^- \to 0$ since $a_n \to 0 \implies b_n \to 0, c_n \to 0$.
\item $\sum a_n^+$ and $\sum a_n^-$ both diverge because they differ by 0 from $\sum b_n, \sum c_n$ respectively.
\end{enumerate}
Set $m_0 = n_0 = -1$. Since $\sum a_n^+$ diverges we may use the well-ordering principle: $\exists m_1 = \min\{k \in \mathbb{N} \;|\; \sum_{n=0}^k a_n^+ > c\}$. Similarly, $\exists n_1 = \min\{k \in \mathbb{N} \;|\; \sum_{n=0}^{m_1} a_n^+ - \sum_{n=0}^k a_n^- < c\}$.

Next, if $m_p$ and $n_p$ are known, we set:
\begin{align*}
m_{p+1} &= \min\left\{k \in \mathbb{N} \;|\; \sum_{l=0}^{p-1} \sum_{j=1+m_l}^{m_l} a_j^+ - \sum_{l=0}^{p-1} \sum_{j=1+n_l}^{n_l} a_j^- + \sum_{j=1+m_p}^k a_j^+ > c\right\}\\
n_{p+1} &= \min\left\{k \in \mathbb{N} \;|\; \sum_{l=0}^{p-1} \sum_{j=1+m_l}^{m_l} a_j^+ - \sum_{l=0}^{p-1} \sum_{j=1+n_l}^{n_l} a_j^- + \sum_{j=1+m_p}^{m_{p+1}} a_j^+ - \sum_{j=1+n_p}^k a_j^- < c\right\}
\end{align*}

Consider the series $(a_1^+ + \cdots + a_{m_1}^+) - (a_1^- + \cdots + a_{n+1}^-) + (a_{1+m_1}^+ + \cdots + a_{m+2}^+) - (a_{1+n_1}^- + \cdots + a_{n_2}^-) + \cdots$. This is clearly a rearrangement of $\sum_{n=0}^\infty a_n$.

Write $A_p = \sum_{l=1+m_p}^{m_{p+1}} a_l^+, A_p^- = \sum_{l=1+n_p}^{n_{p+1}} a_l^-$, and let $S_j$ denote the $j^\text{th}$ partial sum of the rearrangement.\\
By construction, $\limsup_{j \to \infty} S_j = \limsup_{p \to \infty} (\sum_{l=0}^{p+1} A_l^+ - \sum_{l=0}^p A_l^-)$ and\\
$\liminf_{j \to \infty} S_j = \liminf_{p \to \infty} (\sum_{l=0}^p A_l^+ + \sum_{l=0}^p A_l^-)$.

Also, $c < \sum_{l=0}^{p+1} A_l^+ - \sum_{l=0}^p A_l^- < c + a_{m_{p+1}}^+$ and $c - a_{n_{p+1}}^- < \sum_{l=0}^{p+1} A_l^+ - \sum_{l=0}^{p+1} A_l^- < c$.

Thus, by the squeeze lemma, $\lim_{p \to \infty} (\sum_{l=0}^{p+1} A_l^+ - \sum_{l=0}^p A_l^-) = \lim_{p \to \infty} (\sum_{l=0}^p A_l^+ - \sum_{l=0}^p A_l^-) = c$, and so $\lim_{j \to \infty} S_j = c \implies \sum_{n=0}^\infty a_{\gamma(n)} = c$.
\end{quote}

\emph{Remark}: One can also rearrange such that $\sum a_{\gamma(n)} = \pm \infty$.

\newpage

\section{Topology of $\mathbb{R}$}

Our goal in Section 4 is to develop some tools for understanding the ``topology" of $\mathbb{R}$, which is a sort of generalized qualitative geometry.

\subsection{Open and Closed Sets}

\textbf{Definition}:
\begin{quote}\vspace{-0.3cm}
\begin{enumerate}
	\item For $a,b \in \mathbb{R}$ with $a \leq b$, we define:
	\begin{align*}
		(a,b) = \{x \in \mathbb{R} \;|\; a < x < b\} \hspace{1.5cm} [a, b) = \{x \in \mathbb{R} \;|\; a \leq x < b\}\\
		(a,b] = \{x \in \mathbb{R} \;|\; a < x \leq b\} \hspace{1.5cm} [a,b] = \{x \in \mathbb{R} \;|\; a \leq x \leq b\}
	\end{align*}

	\item For $x \in \mathbb{R}$ and $\epsilon > 0$, we set $B(x, \epsilon) = (x-\epsilon, x + \epsilon)$ and $B[x, \epsilon] = [x-\epsilon, x + \epsilon]$.\\
	We call the set $B(x, \epsilon)$ a \emph{neighborhood} of $x$ or a ``ball of radius $\epsilon$ centered at $x$".

	\item A set $E \subseteq \mathbb{R}$ is \emph{open} if $\forall x \in E.\; \exists \epsilon > 0.\; B(x, \epsilon) \subseteq E$.\\
	In other words, every point in $E$ has a neighborhood contained in $E$.
\end{enumerate}
\end{quote}
\emph{Examples}:
\begin{quote}\vspace{-0.3cm}
	\begin{enumerate}
	\item $\varnothing$ is vacuously open.

	\item $\mathbb{R}$ is open because $\forall x \in \mathbb{R}.\; B(x, 1) \subseteq \mathbb{R}$.

	\item If $a < b$ then $(a,b)$ is open.\\
	\emph{Proof}: Fix $x \in (a,b)$ and let $\epsilon = \min\{x-a, b-x\} > 0$. Then $a \leq x - \epsilon < x < x + \epsilon \leq b$ by construction, and $B(x, \epsilon) \subseteq (a,b)$.

	\item If $a < b$ then $[a,b)$ is not open.\\
	\emph{Proof}: For $x = a$ we know that $\forall \epsilon > 0.\; a - \epsilon \notin [a, b)$ and hence $B(a, \epsilon) \not \subseteq [a,b)$.

	\item $[a,b]$ is not open, nor is $(a,b]$ by previous argument.
	\item $E = \{a\}$ is not open.
	\item $E = \{\frac{1}{n} \;|\; n \in \mathbb{N}, n \geq 1\}$ is not open: $\forall \epsilon > 0.\; B(1, \epsilon) \not \subseteq E$.
	\end{enumerate}
\end{quote}

\textbf{Lemma}: If $E_\alpha \subseteq \mathbb{R}$ is open $\forall \alpha \in A$ (some index set), then $\bigcup_{\alpha \in A} E_\alpha$ is open.

\emph{Proof}: Let $x \in \bigcup_{\alpha \in A} E_\alpha$. Then $x \in E_{\alpha_0}$ for some $\alpha_0 \in A$. Since $E_{\alpha_0}$ is open, $\exists \epsilon > 0.\; B(x, \epsilon) \subseteq E_{\alpha_0} \subseteq \bigcup_{\alpha \in A} E_\alpha$.

\end{document}